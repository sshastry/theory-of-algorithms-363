\documentclass[10pt,reqno]{amsart}
\usepackage{amsmath}
\usepackage{amsthm}
\usepackage{amssymb}
\usepackage{graphicx}
\usepackage{mathrsfs}
\usepackage{color}
\usepackage{stmaryrd}
\usepackage{hyperref}
\usepackage{eucal}
\usepackage{fullpage}
\usepackage{algorithm}
\usepackage{algpseudocode}
\usepackage{tikz}
\usetikzlibrary{matrix,arrows}
\tikzset{node distance=2cm, auto}
\usepackage{outlines}
%%%%%%%%%%%%%%%%%%%%%%%%%%%%%%%%%%%%%%%%%%%%%%%%%%%%%%%%%%%%%%%%%%%%%%

\begin{document}

\noindent \textit{Math 363: Theory of Algorithms}

\noindent \textit{Instructor: Sreekar M.~Shastry}

\noindent \textit{Solutions to the Final Examination}

\noindent \textit{2012-04-26-Thu 0900-1200}

\medskip

\begin{itemize}
    \item There are 7 problems. Each problem is worth 5 points. The
        maximum score is 35 points.

\end{itemize}

\medskip

\begin{outline}[enumerate]

\1 Consider the following problems.

\medskip

3-SAT

Independent Set, Set Packing

Vertex Cover, Set Cover

3-D matching, Graph 3-Coloring

Hamilton Cycle, Hamilton Path, Traveling Salesman

Subset Sum, Knapsack

\medskip

(a) Give \emph{precise} definitions for as many of these problems as you can.

(b) Prove two (and no more than two!) polynomial reductions among these
problems.

\medskip
\noindent \emph{Solution.} In chapter 8 of KT.
\medskip

%\1 (a) What is P?
%
%\noindent (b) What is NP?
%
%\noindent (c) What is an NP-complete problem?
%
%\medskip
%\noindent \emph{Solution.} See chapter 8 of KT.
%\medskip

\1 At a summer sports camp, for each sport offered, there should exist one
counselor who has the skills to guide the students in that sport. The camp has
received job applications from $m$ potential counselors, each of whom is
skilled in at least one of the sports offered. Moreover, for each of the $n$
sports there is some applicant qualified in that sport.

Consider the following algorithmic problem: For a given number $k \le m$, is it
possible to hire at most $k$ of the counselors and have at least one qualified
in each of the $n$ sports?

Show that this problem is NP-complete.

\medskip
\noindent \emph{Solution.} Let us recall the Set Cover problem: Given a set $U$
of $n$ elements, a collection $S_1, S_2, \dots, S_m$ of subsets of $U$, and a
number $k$, does there exist a collection of at most $k$ of these sets whose
union is equal to all of $U$?

It is shown on page 472 of KT that this problem is NP-complete.

To solve the problem at hand, it suffices to show that an arbitrary instance of
Set Cover can be reduced to an instance of our sports camp problem. Given an
instance of Set Cover, regard $U$ as a set of sports, and the set $S_i$ as the
set of sports that the $i$th applicant is skilled in. Thus a black box for the
sports camp problem solves Set Cover, as required.

\medskip

\1 The Fibonacci numbers are defined by the following recurrence:
\[ F_0 := 0, F_1 := 1, \text{ and } F_i := F_{i-1}+F_{i-2} \text{ for } i \ge 2.
\]
We may directly convert this mathematical definition into a recursive function
to compute the Fibonacci numbers.

\medskip
\begin{algorithmic}[1]
    \Procedure{Fib}{$n$}
    \If{$n=0$ or $n=1$}
    \State return $n$
    \EndIf
    \If{$n \ge 2$}
    \State return \textsc{Fib}$(n-1)+$\textsc{Fib}$(n-2)$
    \EndIf
    \EndProcedure
\end{algorithmic}
\medskip

It is a basic fact which you make take for granted that if we write $\varphi :=
(1+\sqrt{5})/2$ and $\overline{\varphi} := (1-\sqrt{5})/2$ then $F_i =
(\varphi^i - \overline{\varphi}^i)/\sqrt{5}$.

(a) Show that with the above definition, \textsc{Fib}$(n)$ has running time
$O(\text{\textsc{Fib}}(n))$. Conclude that it requires exponential time.

(b) Using memoization, give a linear time algorithm to compute
$\text{\textsc{Fib}}(n).$

\medskip
\noindent \emph{Solution.} (a) Let $f_n$ be the number of calls to
$\text{\textsc{Fib}}(\cdot)$ made in the course of the computation of
$\text{\textsc{Fib}}(n)$. By this we mean the following: we initialize $f_n:=0$
and as we trace through the execution of $\text{\textsc{Fib}}(n)$, we increment
$f_n$ by one each time we make a call to $\text{\textsc{Fib}}(i)$ for some $i
\le n$.

We claim that $f_n = 2F_n + 1$. The proof proceeds by induction. For $n=0,1$
we have that $f_n = 2\cdot 1 - 1 = 1$. Now suppose that $f_n = 2F_n + 1$. We
compute
\begin{align*}
    f_{n+1} & = f_{n}+ f_{n+1} + 1 & (+1 \text{ is the initial call to \textsc{Fib}}(n+1)) \\
    & = (2F_n -1) + (2F_{n-1} -1) + 1 & (\text{by induction})\\
    & = 2F_{n+1}+1
\end{align*} as required.

The fact that this is exponential follows from the formula $F_i = (\varphi^i -
\overline{\varphi}^i)/\sqrt{5}$.

(b)  The following algorithm runs in time $O(n)$.

\medskip
\begin{algorithmic}[1]
    \State allocate an array memo$[0:N]$, where $N$ is the maximum allowed input
    \State initialization: memo[0] := 0, memo[1] := 1, memo$[k]$ := 0 for $k \ge 2.$
    \Procedure{F}{$n$}
    \If{$n=0$ or $n=1$ or memo$[n] \neq 0$}
    \State return memo$[n]$
    \EndIf
    \State memo[n] := \textsc{F}$(n-1)+$\textsc{F}$(n-2)$
    \State return memo$[n]$
    \EndProcedure
\end{algorithmic}
\medskip

\medskip

\1 There are two types of wrestlers, the Good Guys and the Bad Guys. Between
any pair of wrestlers, they may or may not be a rivalry. Suppose that we have
$n$ professional wrestlers and we have a list of $m$ pairs of wrestlers for
which there are rivalries. Give an $O(n+m)$-time algorithm that determines
whether it is possible to designate some of the wrestlers as Good Guys and the
remainder as Bad Guys such that each rivalry is between a Good Guy and a Bad
Guy. If it is possible to perform such a designation, your algorithm should
produce it.

\medskip
\noindent \emph{Solution.} Examples of wrestlers are Macho Man Randy Savage,
The Undertaker, The Iron Sheikh, and Hulk Hogan. There is a pair (Hulk Hogan,
The Iron Sheikh) indicating a rivalry between the two wrestlers.

Let $G$ be a graph with the wrestlers as nodes and the rivalries as edges.
Perform as many breadth first traversal so as to visit all vertices. Choose a
root node arbitrarily and declare that it corresponds to a Good Guy. Declare to
be Good Guys all wrestlers whose distance from the root is an even integer. The
Bad Guys are those at an odd distance.

Now, we must verify by looping over the edges that each edge goes between a
Good Guy and a Bad Guy. If this is not the case, then we are guaranteed the
existence of the following configuration of edges, where the bottom row
consists nodes at distance $d$ from the root and the top row at distance $d+1$
for some positive integer $d$ (it is possible that $x=y$).

\begin{center}
\begin{tikzpicture}[scale=.9, transform shape]
\tikzstyle{every node} = [circle]
\node (x) at (0, 0) {$x$};
\node (y) at +(0: 1) {$y$};
\node (r) at +(90: 1) {$r$};
\node (s) at +(45: 1.4142135) {$s$};
\foreach \from/\to in {r/s, x/r, y/s}
\draw [-] (\from) -- (\to) node[pos=0.5, sloped, above] {};
\end{tikzpicture}
\end{center}

This picture implies that it is not possible to consistently assign Good Guys
and Bad Guys compatibly with rivalries and we may return ``Not possible.''
Another way to say the same thing is that Good Guys is the color blue and Bad
Guys is the color red. This configuration implies that it is impossible to
color both $x,y$ with one color and both $r,s$ with the other color.

If no edge goes between a Good Guy and a Bad Guy (i.e.~the above loop went
through without hitting an offending edge) we return the sets of Good Guys and
Bad Guys that we constructed earlier.

The BFS takes $O(n+m)$, the assignment of Good and Bad takes $O(n)$, and the
verification loop over edges takes $O(m)$. Thus the complexity is $O(m+n)$, as
required.

\medskip

\1 Consider the problem of making change for $n$ cents using the fewest number
of coins, where we are given a set of possible coin denominations.

(a) Describe a greedy algorithm to make change consisting of quarters, dimes,
nickels, and pennies (of denominations 25, 10, 5, and 1 cents respectively).
Prove that your algorithm yields an optimal solution.

% (b) We modify part (a) as follows. Suppose that the available denominations
% are all powers of a given $c$, i.e.~the denominations are $1, c, c^2, c^3,
% \dots, c^k$ for some integers $c > 1$ and $k \ge 1$. Show that your greedy
% algorithm yields an optimal solution.

(b) Give an example of a set of coin denominations for which the greedy
algorithm does not yield an optimal solution. Your set should include a penny
so that a solution exists for every $n$.

\newcommand{\opt}[1]{\ensuremath{\mathrm{opt}(#1)}}

\medskip
\noindent \emph{Solution.} We first investigate the structure of optimal
solutions.

Let \opt{n} be the optimal solution for making change worth $n$ cents, and
suppose it uses $k$ coins to solve the problem. Suppose that this solution uses
a coin worth $c$. We claim that \opt{n} contains within it \opt{n-c}, and that
the latter uses $k-1$ coins to optimally make change for $n-c$ cents. To prove
the claim, clearly there are at most $k-1$ coins in \opt{n-c} and if there were
fewer than $k-1$, we could use this solution together with the coin of
denomination $c$ to produce a solution of \opt{n} which uses fewer than $k$
coins.

(a) A greedy algorithm to make change using quarters, dimes, nickels, and
pennies is as follows. For a positive real number $x,$ write $[x]$ for the
greatest integer smaller than $x$. We are given $n$ and we must make change for
it.

Use $q := [n/25]$ quarters. Put $n_q := n \mod 25$ (by this we mean the
positive integer which is less than 25 which is congruent to $n_q \mod 25$).

Then use $d := [n_q/10]$ dimes. Put $n_d := n_q \mod 10$.

Then use $k := [n_d/5]$ nickels. Put $n_k := n_d \mod 5$.

Then use $p := n_k$ pennies.

Another, less efficient way to prescribe the same algorithm is as follows. We
want to make change for $n$ cents. If $n=0$ then put $\opt{0} := \varnothing$.
If $n > 0$ then determine the largest coin whose value $c$ is $\le n.$ Give one
such coin and then recursively solve the subproblem of making change for $n-c$
cents.

To prove the correctness of this algorithm, it suffices by what we have already
said to prove the following property, which we will refer to as $P_n$: There
exists an optimal solution for making change for $n$ cents contains a coin
worth $c$ cents where $c$ is the largest coin value such that $c \le n.$
Suppose we are given an optimal solution. If it already contains a coin of
denomination $c,$ we are done. So suppose it does not contain such a coin; we
will transform it into a solution on fewer coins, thus verifying $P_n.$

There are four cases.

(i) $1 \le n < 5 \Rightarrow$ the solution consists of only pennies so $P_n$ is
verified.

(ii) $5 \le n < 10 \Rightarrow$ the solution consists only of pennies since it
has no nickel. Replace five pennies by a nickel to give a solution with fewer
coins.

(iii) $10 \le n < 25 \Rightarrow c = 10.$ By assumption, there is no dime in
the solution, so it contains only nickels and pennies. Some subset of nickels
and pennies must add up to 10 cents (if there are two nickels we are done; if
there is one nickel, use it and five pennies; if there are no nickels, use ten
pennies). Thus we may replace this subset of nickels and pennies by a dime to
give a solution with fewer coins.

(iv) $25 \le n \Rightarrow c=25.$ By supposition, the optimal solution contains
no quarter, only nickels, dimes, and pennies. If it contains three dimes, we
can replace that configuration by one quarter and one nickel for a better
solution. If it contains at most two dimes, then some subset of the dimes,
nickels, and pennies adds up to 25 cents, and we can replace that subset by a
quarter for a better solution.

Regarding the running time, the first description of the algorithm which uses
the greatest integer function and modular arithmetic runs in $O(1)$ because we
only have to do two computations for each of the four denominations
$1,5,10,25$. The recursive version of the algorithm runs in $O(k)$ times where
$k$ is the number of coins used in an optimal solution; since $k \le n$ we
obtain a running time of $O(n)$.

(b) Suppose that $n=30$ and we only have denominations of $1,10,25$
i.e.~pennies, dimes, and quarters but no nickels. The greedy solution gives one
quarter and five pennies for six coins total. The optimal (non-greedy) solution
consists of three dimes.

\medskip

% \1 Consider the problem of a stockbroker trying to sell a large number of
% shares in a company whose stock price has been steadily falling in value. It
% is always hard to predict the right moment to sell a stock, but owning a lot
% of shares in a single company adds an extra complication: the mere act of
% selling many shares in a single day will have an adverse effect on the price.
%
% Let us consider the following simple model. Suppose we have $N$ shares of
% stock in our portfolio which we are tasked to sell. Suppose that the stock
% prices will take the values $p_1, p_2, \dots, p_n$ over the next $n$ days. We
% are given a function $f(\cdot)$ that predicts the effect of large sales: if
% we sell $y$ shares on a single day, it will permanently decrease the price by
% $f(y)$ from that day onward. So, if we sell $y_1$ shares on day 1, we obtain
% a price per share of $p_1 - f(y_1)$ for a total income of $y_1(p_1 -
% f(y_1)).$ Having sold $y_1$ on day 1, we can then sell $y_2$ on day 2 for a
% price per share of $p_2-f(y_1)-f(y_2)$ yielding an income of $y_2 (p_2 -
% f(y_1) - f(y_2))$. And so on. (Note that, as in the day 2 calculation, the
% decreases from earlier days are absorbed into the prices for all later days.)
%
% Design an efficient algorithm which takes the prices $p_1,\dots,p_n$ and the
% function $f(\cdot)$ (given as a list of values $f(1),f(2),\dots,f(N)$ and
% determines the best way to sell $N$ shares by day $n$. In other words, find
% natural numbers $y_1, y_2, \dots, y_n$ such that $N = y_1 + y_2 + \cdots +
% y_n$ and selling $y_i$ shares on day $i$ for $i = 1,\dots, n$ maximizes total
% income.
%
% Assume that the share price $p_i$ is monotonically decreasing and that
% $f(\cdot)$ is monotonically increasing, i.e.~selling a larger number of
% shares causes a larger drop in price.
%
% Your algorithm's running time can have polynomial dependence on $n$ (the
% number of days), $N$ (the number of shares), and $p_1$ (the peak price of the
% stock).
%
% Example. Suppose that $n=3$ and the prices for the three days are 90, 80, and
% 40 and \[ f(y) = \begin{cases}
% 1 & y \le 40,000\\
% 20 & y > 40,000.
% \end{cases}\] We have $N=100,000$ shares in our portfolio. Selling all of
% them on day 1 would yield a price of 70 per share for a total income of 7
% million. On the other hand, selling 40,000 shares on day 1 gives a price of
% 89 per share and selling the remaining 60,000 shares on day 2 yields a price
% of 59 per share, for a total income of 7.1 million.

\1 Let us recall the Subset Sum problem. Given a set of natural numbers
$\{w_1,w_2,\dots,w_n\}$ and a target number $W,$ is there a subset of
\(\{w_i\}\) that adds up precisely to $W$?

It is shown in chapter 8 of the text that this is an NP-complete problem.

On the other hand, it is shown in chapter 6 (using dynamic programming) that
there is an algorithm with running time $O(nW)$ that solves this problem

Why does this not imply that P=NP?

\medskip
\noindent \emph{Solution.} This is explained on page 491 of KT.
\medskip

\1 Given a sequence $A := \{a_1,a_2,\dots,a_n\}$ of distinct real numbers, find
the size of the largest subset such that for every $i < j$, $a_i < a_j.$ In
other words, find the length of the longest increasing subsequence of $A.$ Your
algorithm should have $O(n^2)$ or better running time. (Hint: use dynamic
programming.)

\medskip
\noindent \emph{Solution.} Let \opt{k} be the length of the longest increasing
subsequence of $A$ subject to the constraint that the sequence ends with $a_k$.
Since the longest increasing subsequence of $A$ must end on some element of
$A$, we can find its length by finding the maximum value of opt. It remains to
actually compute the function \opt{\cdot}.

Put \[ S_k := \{ i : i < k \text{ and } a_i < a_k\}.\] If $S_k = \varnothing$
then all elements that come before $a_k$ are greater than it and thus $\opt{k}
= 1$ in this case. Otherwise, $S_k \neq \varnothing$ and we have the recursion
\[\opt{k} = \max (\opt{j} : j \in S_k) + 1.\] In words, the longest subsequence
sequence ending with $a_k$ consists of the element $a_k$ appended to the
longest subsequence ending with some $a_j$ where $j < k$ and $a_j < a_k$. The
algorithm goes as follows.

\begin{algorithmic}[1]
    \Procedure{LongestIncreasingSubsequence}{$A$}
    \State Let $n$ be the length of $A$
    \State Let opt be an array of length $n$, initially all zero
    \For{$k = 0,\dots, n$}
    \State $\mathrm{max} := 0$
    \For{$j = 0,\dots, k$}
    \If{$A[k] > A[j]$} \emph{// this restricts us to $S_k$}
    \If{$\mathrm{opt}[j] > \mathrm{max}$}
    \State max := opt$[j]$ \emph{// find $\max (\opt{j} : j \in S_k)$}
    \EndIf
    \EndIf
    \EndFor
    \State $\mathrm{opt}[k] := \mathrm{max} + 1$
    \EndFor
    \State return $\max\{\mathrm{opt}[i] : i = 0,\dots,n\}$
    \EndProcedure
\end{algorithmic}

\medskip
\medskip

\end{outline}
\end{document}
